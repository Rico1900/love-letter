\documentclass[12pt]{article}

\usepackage[margin=1in]{geometry}
\usepackage[UTF8,fontset=none]{ctex}
\usepackage{microtype}
\usepackage{graphicx}
\usepackage{xcolor}
\usepackage{tikz}
\usepackage{pgfornament}
\usepackage{setspace}
\usepackage{fancyhdr}

\setlength{\parindent}{0pt}
\setlength{\parskip}{0.8\baselineskip}
\onehalfspacing

\setCJKmainfont{Songti SC}
\setCJKsansfont{Heiti SC}
\setCJKmonofont{Heiti SC}
\newCJKfontfamily\kaishu{Heiti SC}

\pagestyle{fancy}
\fancyhf{}
\fancyfoot[C]{%
  \ifnum\value{page}=1 220%
  \else\ifnum\value{page}=2 284%
  \else\thepage%
  \fi\fi
}
\renewcommand{\headrulewidth}{0pt}

\definecolor{ink}{HTML}{2B2B2B}
\definecolor{accent}{HTML}{6B1F2B}
\pagecolor{white}
\color{ink}

% Decorative rules (no external assets)
\newcommand{\ornamentrule}{%
  \par\noindent
  \begin{center}
    \color{accent}\pgfornament[width=0.72\linewidth]{88}
  \end{center}
  \par
}

\newcommand{\lettertitle}[1]{%
  \begin{center}
    \vspace*{0.5em}
    {\color{accent}\pgfornament[width=0.55\linewidth]{75}\par}
    \vspace{0.6em}
    {\Large\bfseries #1\par}
    \vspace{0.6em}
    {\color{accent}\pgfornament[width=0.55\linewidth]{75}\par}
  \end{center}
  \vspace{0.6em}
}

\begin{document}

\lettertitle{致文珺}

\begin{flushright}
{\color{accent}\kaishu 我心匪石,不可转也。我心匪席,不可卷也。——《诗经·邶风·柏舟》}
\end{flushright}

\ornamentrule

{\kaishu My dearest \emph{P-Body},}

思绪回转,忽已是甲午(二〇一四)旧年。

彼时日之所盼,唯与卿相游于太虚矩阵。当年网路迟滞,联机多艰,你我只通声息,未见形影。吾常托辞视频侵耗带宽,恐生卡顿。以信息论观之,此言固然非虚;然究其本心,实为年少羞涩未褪。恰如新酿之葡萄酒,未经岁月沉放,卿与吾相伴,想必尝尽了粗粝之单宁。

忆昔方块世界(Minecraft),琐细已不可考。想来无非同掘细沙以炼明窗,共植麦穗以焙干粮。夜半流浪,纵遭骷髅乱箭穿身,亦赖卿之面包回血续命。吾也曾铺设延绵之铁轨,欲载卿呼啸而行,穷尽此间天地。及至「饥荒」之境,吾竟未知饥荒为何物。卿极善囤聚物资,俨然一只勤勉之仓鼠。吾与仓鼠同室而居,但食仓鼠之余粮,又何馁之有?又及「传送门」,叠嶂迷宫,目眩神迷。卿虽不适此等倒悬之境,然若无卿相伴,愚钝如我亦断难成事。

前路关卡重重,你我之征途远未告罄。今吾与卿既育得一子Gabriel,吾侪三人之新卷,方才徐徐展开。此后当周游列国,啖四海之西瓜,品异域之咖啡;亦欲溯流时光,效中世纪之起居与教化。尤堪玩味者,乃于此人工智能勃兴之秋,你我偏要同游幽邃之灵域,叩问天主造化吾辈之深意。

\vspace{1em}
\begin{center}
\includegraphics[width=0.5\linewidth,keepaspectratio]{couple.png}
\end{center}

\ornamentrule

\begin{flushright}
I carry your heart with me,\\[0.5em]
{\Large\kaishu Atlas}
\end{flushright}

\vfill
\begin{center}
{\color{accent}\pgfornament[width=0.78\linewidth]{87}}
\end{center}

\end{document}
